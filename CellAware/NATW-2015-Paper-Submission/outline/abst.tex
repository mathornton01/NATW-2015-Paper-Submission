    \begin{comment}

    This paper outlines our current progress in the search for intelligent targeting methods for a class of faults known as ``Cell-Aware Type Faults''. We begin with a brief introduction to cell-aware faults and a description of how they differ from the cell-aware type faults that we are examining in this endeavor. We then describe our over-arching research goal: To find an attribute that can predict whether or not cell-aware faults can cause serious problems during functional usage. We then present our research methodology beginning with the generation of a describing-fault-model for cell-aware type faults, and an analysis of potential attributes that can be used to classify and characterize cell-aware type faults. Our findings are discussed, and the respective benefits to manufacturing test are discussed.


    \end{comment}

    The ``Cell-Aware'' fault model does an excellent job of expressing potential in-gate defects. Unfortunately some circuits contain too many potential cell-aware faults to test for with the limited resources available to test engineers. To avoid detailed analog modeling, and fault extraction we modeled cell-aware faults using a methodology wherein stuck-at ATPG was the primary fault modeling procedure used.  We differentiate these faults from their analog counterparts by deeming them ``Cell-Aware-\textit{Type}'' faults
    To determine the most important faults to target, we extracted the ``mandatory conditions'' for detection for each potential fault in two test circuits, and performed functional simulation to determine how many times these mandatory conditions were met.
    We then analyzed the correlation between mandatory condition counts for faults, and the corresponding faults detection during goodstate simulation. 
    Our results include the discovery that mandatory conditions are a good predictor of a fault's relative importance in a circuit whose intended function is known, provided that the potential state space is sufficiently explored during functional operation. 