    The abstract concept of a cell-aware fault is best described as any fault that is caused by a certain defect within a standard cell. Meaning that the fault specifically models a problem within the transistor array of the cell. Because this describes exactly every defect from manufacturing view, the fault model can be thought of more effectively as a model that represents faults that cannot be explained by some other traditional fault model such as the stuck at or bridging fault model. The cell-aware model and it's testing framework as well as ATPG methods were introduced by Hapke et al. \cite{5355741} \cite{Hapke} \cite{6401533}. The cell-aware fault model pays particular attention to the contents of the standard cell library which is used in the design of a circuit. The promise of using this fault model was explored in more detail by Rajski et al. \cite{5227030}. The effectiveness of the cell-aware fault model for detecting faults, and a comparison to the small delay defect fault model was explored by Fan Yang et al. \cite{6979084}. A case study regarding the usefulness of the cell-aware fault model on diagnosing a micro-controller was done by Prabhu et al. \cite{6847826}.

    In conjunction with the research discussed in this paper we discovered that n-detect for forcing multiple detections in certain ATPG schemes wouldn't work for cell-aware type faults. Consequently the definition of a cell-aware type fault in contrast to a pure cell-aware fault was discussed in a previous paper we wrote. \cite{6875445} The difference between what we have been referring to as a cell-aware type fault and a pure cell-aware fault arises in the derivation of each. As above mentioned, the cell-aware fault model can be viewed two ways that are essentially the same. The first is as an internal defect something tangible within the transistor array, Cell-aware faults can be derived by examining the electrical connections, and relationships. The second is as a fault that cannot be described by another fault model, and would thus not be tested for in ATPG targeting a specific type of fault. viewing the cell-aware fault by this second means is simply not the standard meaning for one, and we have hence taken to referring to them as cell-aware type faults. The definition of, and creation of the cell-aware type fault model is where the discussion of our methodology begins.
