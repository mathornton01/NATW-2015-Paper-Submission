\begin{comment}


    The abstract concept of a cell-aware fault is best described as any fault that is caused by a certain defect within a standard cell. Meaning that the fault specifically models a problem within the transistor array of the cell. Because this describes exactly every defect from manufacturing view, the fault model can be thought of more effectively as a model that represents faults that cannot be explained by some other traditional fault model such as the stuck at or bridging fault model. The cell-aware model and it's testing framework as well as ATPG methods were introduced by Hapke et al. \cite{5355741} \cite{Hapke} \cite{6401533}. The cell-aware fault model pays particular attention to the contents of the standard cell library which is used in the design of a circuit. The promise of using this fault model was explored in more detail by Rajski et al. \cite{5227030}. The effectiveness of the cell-aware fault model for detecting faults, and a comparison to the small delay defect fault model was explored by Fan Yang et al. \cite{6979084}. A case study regarding the usefulness of the cell-aware fault model on diagnosing a micro-controller was done by Prabhu et al. \cite{6847826}.

    In conjunction with the research discussed in this paper we discovered that n-detect for forcing multiple detections in certain ATPG schemes wouldn't work for cell-aware type faults. Consequently the definition of a cell-aware type fault in contrast to a pure cell-aware fault was discussed in a previous paper we wrote. \cite{6875445} The difference between what we have been referring to as a cell-aware type fault and a pure cell-aware fault arises in the derivation of each. As above mentioned, the cell-aware fault model can be viewed two ways that are essentially the same. The first is as an internal defect something tangible within the transistor array, Cell-aware faults can be derived by examining the electrical connections, and relationships. The second is as a fault that cannot be described by another fault model, and would thus not be tested for in ATPG targeting a specific type of fault. viewing the cell-aware fault by this second means is simply not the standard meaning for one, and we have hence taken to referring to them as cell-aware type faults. The definition of, and creation of the cell-aware type fault model is where the discussion of our methodology begins.

\end{comment}
The cell-aware fault model is central to this research. 
To place it in context, we begin with a discussion of its history. 
The cell-aware fault model and its testing framework were defined by Hapke et al. \cite{5355741}, although previous work along the same lines was done by Sharma et al.\cite{4437604}. 
Hapke has done further work with the model by creating a tutorial for its use in industry\cite{Hapke}, applying it to a 32-nm laptop processor in a case study\cite{6401533},  applying it to automotive parts in another case study\cite{6847814}, creating a new testing procedure for it \cite{6879635}, showing industrial results \cite{5783604}, and extending it to allow for detection of some internal small delay defects\cite{6139151}. 
Fan Yang et al. compared results from the use of both the cell-aware and small delay defect fault model in \cite{6979084}.

The method by which we perform functional simulation was first proposed as a way to determine fault coverage information at ITC in 2011 by Shi et. al\cite{6139146}. 
This work itself was an extension of an earlier paper that described targeting very difficult stuck-at faults\cite{4700614}. 
Our research is different from this previous work due to the use of the cell-aware fault model and the non-exclusive targeting of only hard stuck-at faults.
On the contrary, in this research many faults may be easy to detect.

 Previously we have shown that some cell-aware-type faults are resistant to n-detect, especially when a test set is optimized for multiple detections of stuck at faults with few patterns. 
 In the same paper, we discussed using cell-aware-type top-off patterns as a means to prioritize cell-aware-type faults when testing resources are limited.\cite{6875445}. 

